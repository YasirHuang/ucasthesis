\chapter{基于跨模态文本重构的神经机器翻译}
% 本文应该侧重的是:
%    明确的视觉信息融合方式;
%    短语和词级别的视觉信息融合;
%    模型不需要对视觉信息敏感,因为视觉信息直接作用到目标上了

% 摘要
% 原文摘要:现有多模态机器翻译将视觉信息以全局特征的方式作为固定上下文输入,或者利用注意力机制对图像建立动态上下文作为文本内容的补充信息。然而这种隐含式的视觉信息融合方法对于分析图像如何起到作用以及为什么会产生作用带来了很大的挑战。针对此问题,提出了一种实体级的跨模态信息融合方法,显式地将文本中实体替换为对应图像中的视觉目标,并利用翻译模型来重建出完整的原文本,最后通过多任务的方式将以上重建任务与翻译任务相结合。实验表明,该方法在有效的提升了实体词的翻译质量,译文质量得到了显著的提升。
% 
%现有融入图片信息的神经机器翻译将视觉信息以全局特征的方式作为固定上下文输入,或者利用注意力机制对图像建立动态上下文作为文本内容的补充信息。然而这种隐含式的视觉信息融合方法对于分析图像如何起到作用以及为什么会产生作用带来了很大的挑战。针对此问题,提出了一种实体级的跨模态信息融合方法,显式地将文本中实体替换为对应图像中的视觉目标,并利用翻译模型来重建出完整的原文本,最后通过多任务的方式将以上重建任务与翻译任务相结合。实验表明,该方法在有效的提升了实体词的翻译质量,译文质量得到了显著的提升。

% 神经机器翻译中融合图片信息的问题:普遍范式中,句子与图片的信息融合是隐式的,即无法准确得知视觉信息作用到哪些词上或对句子的表示带来了哪些影响,因此,为了探索是否能够以明确的方式融合视觉信息,明确方式融入视觉信息是否对翻译有效,本章提出了一种在名词短语或名词的粒度上融合图片中的视觉目标信息,并进行跨模态文本重构的重构任务。再通过多任务训练的方式,将重构任务所优化的模型参数与翻译任务共享,从而达到提升翻译质量的目的。本章设置了进一步的分析实验,实验结果表明本章所提方法通过有效的提升实体词的翻译质量,从而提升了翻译准确率。

% 多模态的好处
采用分布式向量表示的神经翻译模型和图像处理基础模型极大地便利了以句子为基础语义单位的跨模态信息融合,
% 研究现状
例如,将视觉信息以全局特征的方式作为固定上下文输入到神经翻译模型中,或者利用注意力机制对图像建立动态上下文作为文本内容的补充信息。
% 存在的问题
然而句子与图片的信息融合是隐式的,即无法准确得知视觉信息作用到哪些词上或对句子的表示带来了哪些影响。
% 本章提出的方法
因此,为了探索是否能够以明确的方式融合视觉信息,以及明确方式融入视觉信息是否能够带来稳定的翻译质量提升,本章提出了一种在名词短语或名词的粒度上融合图片中的视觉目标信息,并进行跨模态文本重构的重构任务。通过多任务训练的方式,将重构任务所优化的模型参数与翻译任务共享,从而达到提升翻译质量的目的。
% 进一步的分析结果
本章设置了进一步的分析实验,实验结果表明本章所提方法通过有效的提升实体词的翻译质量,从而提升了翻译准确率。

\section{引言}
% 背景:
得益于神经网络方法的快速发展,自然语言处理和计算机视觉等领域均得到了飞速的进步,同时也为跨领域的信息融合带来了可能。在神经机器翻译中融合图片信息就是其中一种跨模态信息融合方法。跨模态的信息融合是为了解决一些在翻译中难以解决的问题,例如文本中存在歧义,表达不完整,文本表示欠拟合等问题。图片中往往包含着更丰富的语义信息。因此,如何将图片信息融合到翻译中成为了研究者们所关注的问题,相关方法也层出不穷。

% 相关研究:
目前已有的研究主要关注如何将整个图片中的信息融合到神经机器翻译模型的翻译过程中。由于图像分类任务所使用的卷积神经网络,如Resnet-50\cite{32_DBLP:conf/cvpr/HeZRS16},取得了非常好的效果,因此多数研究这尝试将预训练的卷积神经网络与神经翻译模型相结合。例如部分研究将图片经过卷积神经网络提取得到全局特征后,将其作为一个包含完整的上下文信息的特征向量以各种方式输入到翻译模型中,从而完善翻译模型在编码过程所获得的语义表征或为解码过程提供外源信息以供参考\cite{52_DBLP:journals/corr/ElliottFH15,18_DBLP:conf/emnlp/CalixtoL17,22_li-etal-2021-vision,20_wu-etal-2021-good}。
文献\cite{36_calixto-etal-2017-doubly,47_DBLP:conf/wmt/LibovickyHM18}尝试利用图片局部特征与注意力机制配合获图片的局部动态上下文,以达到在解码过程中融合与当前解码步骤最相关的视觉信息。此类方法中,图片作为一个句子级别的语义信息提供者作用到翻译过程中,试图使翻译模型最大化地利用图片中所包含的信息。
%因视觉信息与文本信息的融合与利用为通过模型学习的方式,而难以明确图片信息在翻译中的作用过程,本文称此类方法为隐式跨模态信息融合方法。

% 存在的问题:
然而,这些方法均以隐式的方式将图片信息与文本信息相结合,这些融合图片信息的神经机器翻译模型是如何应用视觉信息以及为何输入图片能够使得翻译性能提升都是不明确的,本文中,我们称此类方法为隐式跨模态信息融合方法。更重要的是,部分方法存在翻译质量提升不稳定的情况,不同的模型实现方案或不同的模型初始化条件下模型的翻译性能表现差异较大。尽管在句子级语义单位可以最大化地利用图片中所携带的视觉信息,但神经翻译模型似乎不能很顺利地将其利用。

% 本章提出方法
为此,本章提出了一种明确的图片信息融合方法。文献\cite{53_caglayan-etal-2019-probing}发现,当文本中部分内容或信息缺失,尤其是实体词缺失时,神经翻译模型对视觉信息最敏感。为了将图片中的视觉信息明确并有效地融合到模型中,本文以实体为语义单元融合图片信息,其中实体以短语实体或名词实体两种粒度呈现。区别于文献\cite{53_caglayan-etal-2019-probing}通过创造的文本内容缺失环境使隐式跨模态信息融合有效,本文采用显式跨模态信息融合方法,直接利用视觉特征替换实体作为模型输入,规避了神经翻译模型对视觉信息不敏感的问题。为了实现跨模态的信息融合,本文设计了一个重构模型来实现多模态输入到纯文本输出的任务。该模型以图片的视觉目标联合实体被替换的源语言文本为输入,以原文或译文为输出,通过端到端的学习将视觉目标中的视觉信息融合到实体的表示中。最后,本文利用了多任务学习方法,将跨模态文本重构任务与翻译任务相结合。通过多种参数共享机制,使翻译模型能够充分利用到重构任务重所学习到的视觉信息,从而提升翻译质量。

本章主要贡献如下:

(1)本章提出了一种基于跨模态文本重构的神经机器翻译方法(neural machine translation based on cross-modal text reconstruction, CTR-NMT), 并且将所提方法应用到了RNMT和Transformer两种框架。本章详细对比了多种参数共享方案,在推断(inference)时没有额外的图片输入的情况下,达到了与其它模型可比的翻译准确率。与纯文本基线模型相比,有着稳定的翻译质量的提升。

(2)本章提出了一种明确的融合图片信息的方法,即显式跨模态信息融合法。将图片中的视觉目标信息明确的作用到在文本中相对应的实体上,并在基于短语和基于词两种粒度的跨模态融合方案进行了实验对比,发现相比于短语实体,在词实体上融合图片信息效果更佳。

(3)本章对视觉目标的作用对象进行了分析,并发现本章所提的明确的图片信息作用方法,使名词实体的翻译准确率得到了更进一步的提升,从而提升了神经机器翻译模型的翻译质量。

\section{相关工作}

% 相关工作分类:
%     传统方法:句子级融合法,隐式法,(Feature-level Incongruence Reduction for Multimodal Translation,直接使用视觉特征不适合当前模型)
%     重构法:文本重构(Neural Machine Translation with Reconstruction,
%            图像重构(mywork2)(Imagination,Adversarial reconstruction for Multi-modal Machine Translation
%     实体替换法:主要用于分析
%     视觉目标法:
%     预训练模型法:包含纯文本预训练模型,毕竟消融实验中有类似的

% 这里可能需要提一下“退化文本”的概念
\section{实体对齐方法}
\label{sec:3_entity_extraction}
% 我们需要的是什么样的视觉特征,以及为什么
本章所提方法是针对名词短语或名词融合图片中的视觉信息。也有相关工作尝试以词为粒度融合视觉信息,例如文献\cite{20_wu-etal-2021-good,22_li-etal-2021-vision}均采用了门控机制来控制图片的全局特征与输入序列中每个单词的隐层表示的融合。然而,这种方式为模型所提供的图片信息依旧是不明确的。为此,我们选择将图片中与名词或短语相对应的视觉目标提取出来再进行信息的融合。该过程涉及到如何将句子中的名词或短语与图片中的视觉目标对齐。图\ref{fig:3_entity_extraction}为本文的解决方案。
\begin{figure}[!htbp]
    \centering
    \includegraphics{Img/fig_3_entity_extraction.pdf}
    \bicaption{实体提取与对齐方法}{Entity extraction and alignment methods}
    \label{fig:3_entity_extraction}
\end{figure}

图\ref{fig:3_entity_extraction}a展示了名词短语的提取过程。本文采用成熟的spaCy\footnote{https://spacy.io/}工具提取源语言句子中的名词短语。这样再通过对句子的词性标注即可获得所提取的短语中的名词。例如,将句子“A man with a hat.”输入到短语提取工具中,可得到名词短语“A man”和“a hat”,其中“man”和“hat”就是本文需要的名词。因为考虑到句子中仅名词或名词短语在图片中有着直接对应的视觉目标,所以在该方法中不考虑源语言句子中其它类型的短语。

图\ref{fig:3_entity_extraction}b展示了图片中视觉特征的提取过程。在得到了文本中的名词短语后,就可以根据这些短语提取图片中对应的区域作为视觉目标。为此,本文采用了文献\cite{24_DBLP:conf/iccv/YangGWHYL19}提出的单步视觉目标提取法(one-stage visual grounding)。该方法的输入是一张图片和一个短语,模型的输出为短语所对应的图片区域。简化工作流程为,利用预训练的语言模型(如BERT\cite{25_DBLP:conf/naacl/DevlinCLT19})和预训练的图片编码器(如Darknet-53\cite{26_DBLP:journals/corr/abs-1804-02767})分别为短语和图片提取语义特征向量和图片栅格特征,然后将语义特征向量与栅格特征中每个栅格的特征相融合(如拼接)。将融合的特征送入检测模块判断栅格中哪些位置是属于输入的短语,最后拼接成一块完整的区域。例如,将“A man”和完整的图片输入到模型中,提取出来的就是图片中“男人”所对应的区域。

经过以上过程,就可得到源语言句子中的名词、名词短语、视觉目标以及它们之间的对应关系。
\section{方法描述}

本节首先介绍句子和图片的输入方式,来解释如何做到明确的视觉信息作用方法。然后介绍本章设计的文本重构模型是如何工作的。最后介绍如何利用文本重构模型帮助提升翻译模型的翻译质量。

\subsection{实体替换的多模态序列}
\label{sec:3_entity_replacement}

\begin{figure}[!htbp]
    \centering
    \includegraphics{Img/fig_3_example.pdf}
    \bicaption{词实体与短语实体替换示例}{An example of the word entity and the phrase entity}
    \label{fig:3_example}
\end{figure}
本章所提的句子-图片融合方法将作用于句子中的实体,为此我们以两种语义粒度定义文本实体:

(1){\sffamily 短语实体:}一个短语实体是一个视觉可描述短语,其能够完整的描述一个视觉目标图片。例如,图\ref{fig:3_example}中红框中的人“<E1>”在句子$X_1$中被描述为“A man”。其中,“man”是视觉目标,“A”为“man”的数量修饰词,两个词对于所描述的视觉目标均是有明确意义的。

(2){\sffamily 词实体:}一个词实体是短语实体中的名词实体。对于不同的描述者,视觉目标图像可以被描述为不同的短语。如图\ref{fig:3_example}中,同样是视觉目标“<E1>”,在句子$X_2$中则被描述为“Man”,对视觉目标“<E2>”,$X_1$中描述为“a hat”,在$X_2$中则使用了修饰词“orange hat”。为了消除修饰词所带来的影响,在此方案中只考虑名词作为文本实体。

本章所提的明确的图片信息融合方法,就是将源语言句子中的文本实体直接替换为图片中与文本实体相对应的视觉目标。
由于存在两种文本实体,因此本文设置两种针对句子-图片融合方法的替换规则:短语级替换规则(phrase-level replacement rule,PRR)和词级替换规则(word-level replacement rule,WRR)。在PRR中,将短语内部的所有词逐个用该短语所对应的视觉目标图像替换。例如图\ref{fig:3_example}中,$Z_{1,p}$中的“A”和“man”均被替换为“<E0>”对应的视觉目标。WRR与上述方法相似。如图\ref{fig:3_example}中$Z_{1,w}$仅名词部分(“man”,“hat”)被替换为对应的视觉目标(“<E1>”,“<E2>”)。最后的输入为该退化的句子与视觉目标图像的混合序列。

\subsection{文本重构模型}
\label{sec:3_sentence_reconstruction}
\begin{figure}[!htbp]
    \centering
    \includegraphics[scale=1.0]{Img/fig_3_reconstruction_model.pdf}
    \bicaption{融合视觉目标信息的文本重构方法}{Text reconstruction by fusing visual object information}
    \label{fig:3_reconstruction_model}
\end{figure}
为了充分利用文本模态和视觉模态的信息,本章设计了文本重构模型使显式跨模态信息融合方法有效。如图\ref{fig:3_reconstruction_model}所示,左侧为利用图\ref{fig:3_entity_extraction}中的方法提取出视觉目标后,再通过预训练的卷积神经网络提取得视觉目标的全局特征,并利用一个前馈神经网络(feedforward neural network, FNN)映射到所需要的维度。图\ref{fig:3_reconstruction_model}右侧的 “文本重构模型”是一个与“翻译模型”相似的序列到序列的生成模型。其多模态输入序列$Z_{w}$是利用WRR得到的退化文本与视觉目标图像的混合序列。该序列的向量表示为$\{\Vector{e_1,v_2,e_3,e_4,v_5,e_6}\}$。重建任务负责从$Z_{w}$重构回原始文本$X=\{x_1,x_2,…,x_N\}$。模型因此可以分别在编码阶段的视觉特征空间和在解码阶段从语言特征空间学习到实体信息。重建模型的训练方式为最小化以下对数似然方程:
\begin{equation}
    \mathcal{L}_R(\theta, \psi)=-\sum_i^N \log p(x_i|x_{<i},Z)
    \label{eq:3_reconstruction_x}
\end{equation}
其中$Z$可以是使用WRR的$Z_{w}$或是采用了PRR的$Z_{p}$,$\theta$为重构模型的编码器参数,$\psi$为重构模型的解码器参数。
考虑到重构目标语言$Y=\{y_1,y_2,\cdots,y_M\}$也是一种可行方案。此时,目标函数调整为:
\begin{equation}
    \mathcal{L}_R(\theta, \psi)=-\sum_j^M \log p(y_j|y_{<j},Z)
    \label{eq:3_reconstruction_y}
\end{equation}

\subsection{与机器翻译相结合}
\label{sec:3_multitask}
图\ref{fig:3_reconstruction_model}中所展示的重构模型与一般的神经机器翻译模型相似,都是基于编码器-解码器结构的端到端生成模型。其中翻译模型的目标函数也与公式\ref{eq:3_reconstruction_x}和公式\ref{eq:3_reconstruction_y}相近:
\begin{equation}
    \mathcal{L}_T(\theta, \phi)=-\sum_j^M \log p(y_j|y_{<j},X)
    \label{eq:3_translation}
\end{equation}
其中$\phi$为神经翻译模型的解码器参数。为了结合重构任务和翻译任务,本章按照\tcite{elliott2017imagination}的方式将两者的目标函数结合:
\begin{equation}
    \mathcal{L}(\theta, \psi, \phi)=\omega L_T(\theta, \phi) + (1-\omega)L_R(\theta, \psi)
    \label{eq:3_combine_sr}
\end{equation}
其中$\omega$是调节多任务训练比例的超参数,即当前批数据用于更新翻译模型参数的概率。相对应的,用于更新文本重构模型参数的概率为$1-\omega$。当前的多任务学习通过共享编码器参数来达到知识迁移的目的。

\subsection{参数共享策略}
\label{sec:3_parameter_sharing}
% 
\begin{figure}[!htbp]
    \centering
    \begin{subfigure}[b]{0.45\textwidth}
      \includegraphics[scale=0.7]{Img/fig_3_sr.pdf}
      \caption{源-独享}
      \label{fig:3_sr}
    \end{subfigure}%
    ~% add desired spacing
    \begin{subfigure}[b]{0.3\textwidth}
      \includegraphics[scale=0.7]{Img/fig_3_ss.pdf}
      \caption{源-共享}
      \label{fig:3_ss}
    \end{subfigure}
    \\% line break
    \begin{subfigure}[b]{0.3\textwidth}
      \includegraphics[scale=0.7]{Img/fig_3_t.pdf}
      \caption{目-共享}
      \label{fig:3_t}
    \end{subfigure}%
    \bicaption{重构模型与翻译模型的参数共享方案}{Parameter sharing schemes between reconstruction model and translation model}
    \label{fig:4_fidelity}
\end{figure}
正如前面章节所提到,重构模型与翻译模型在结构上有高度的相似性,因此可以利用参数共享的方法将重构模型学习到的视觉信息融合到翻译模型中,从而达到提升翻译质量的目的。因此,由编码器-解码器结构的灵活性与两种重构目标可组合得到多种参数共享方案。本文采用了以下三种参数共享方案。

(1){\sffamily 独享解码器重构源语言}
\begin{figure}[!htbp]
    \centering
    \includegraphics[scale=1]{Img/fig_3_sr.pdf}
    \bicaption{独享解码器重构源语言“源-独享”}{Reconstruction for source language with respective decoder}
    \label{fig:3_sr}
\end{figure}

由于视觉信息要在编码的过程中学习特征表示,因此在所有的参数共享策略中,都要包含共享编码器。解码器的参数共享则是可选的。在利用独享解码器重建源语言的策略中,为源语言和目标语言设立独立的解码器,即重构模型和翻译模型的编码器是共享的,解码器参数是独享的。联合目标函数如公式\ref{eq:3_combine_sr},在后面的实验中,本文用“源-独享”表示应用该策略。

(2){\sffamily 共享解码器重构源语言}
\begin{figure}[!htbp]
    \centering
    \includegraphics[scale=1]{Img/fig_3_ss.pdf}
    \bicaption{共享解码器重构源语言“源-共享”}{Reconstruction for source language with shared decoder}
    \label{fig:3_ss}
\end{figure}

通过融合源语言和目标语言的词表,以及在编码器和解码器之间共享词嵌入层的方式,可以实现重构模型和翻译模型的共享解码器的设置。同时,需要在解码过程目标语言句子输入时,在句首引入一个“识别词”(例如,“<en\_sos>”表示解码到英语,“<de\_sos>”表示解码到德语)用于表示当前解码器用于重构到源语言还是翻译到目标语言。在该设置中,公式\ref{eq:3_combine_sr}中有$\psi=\phi$。这表明该策略中翻译模型和重建模型是全参数共享的。因此,将联合目标函数调整为:
\begin{equation}
    \mathcal{L}(\theta, \phi)=\omega \mathcal{L}_T(\theta, \phi) + (1-\omega)\mathcal{L}_R(\theta, \phi)
    \label{eq:3_combine_ss}
\end{equation}
本文将用“源-共享”代表该参数共享策略。

(3){\sffamily 共享解码器重构目标语言}
\begin{figure}[!htbp]
    \centering
    \includegraphics[scale=1]{Img/fig_3_t.pdf}
    \bicaption{共享解码器重构目标语言“目-共享”}{Reconstruction for target language with shared decoder}
    \label{fig:3_t}
\end{figure}

不同于重构源语言,解码器的参数共享策略在重建目标语言的方向上则很容易实现。在该策略中,重构任务的工作方式与一般的MMT模型相似,区别在于输入的文本是退化文本。并且其联合目标函数与公式\ref{eq:3_combine_ss}相同。本文将用“目-共享”来表示该参数共享策略。
\section{实验设置}
为了验证所提方法的有效性,本章在基于RNN和基于Transformer的两种模型上进行了测试。本章选择了融合图片信息的神经机器翻译中最常用的Multi30K\cite{43_elliott-etal-2016-multi30k}数据集的英译德翻译对上进行实验。

\subsection{实验数据}
本文使用了英-德图像描述的Multi30K翻译数据集。该数据集每张图片配有一个英文句子和与其对应的德文翻译。该数据集共包含31014张图片,并划分为训练集、验证集和测试集三部分,对应的图片数量分别为:29000,1014和1000。另外,本章还采用了WMT17发布的新测试机,包含Multi30K 2017测试集和针对歧义词的ambiguous MSCOCO 2017测试集。这两个测试集同样是每张图片配有一个英文句子和德文翻译。本文采用Moses SMT\cite{44_koehn-etal-2007-moses}工具包对数据进行分词(tokenization)和归一化(normalization)处理。为了防止对视觉实体与文本实体对应关系的破坏,本文并没有采用双节编码技术\cite{27_sennrich-etal-2016-neural}(byte pair encoding, BPE)或WordPiece\cite{28_DBLP:journals/corr/WuSCLNMKCGMKSJL16}进行分词操作。经过以上预处理后,得到英文词表大小为10209,德文词表为18674,合并后的双语词表大小为27226。

\subsection{实体提取}
实体提取采用的是\ref{sec:entity_extraction}节所介绍的方法。其中名词和名词短语提取模块采用的是spaCy\footnote{https://spacy.io/}提供的提取名词短语和词性标注功能。在采用词级替换规则时,所替换的名词占训练集和测试集单词统计量的32.6\%。采用短语级替换规则时,提取的名词短语占单词统计量的45.1\%。同时,我们观察了在词级和短语级替换规则中,所涉及单词的词频情况。两种替换规则下,所涉及单词的词频中位数都是2,这意味着这些词中多数都是低频词。提前视觉实体采用的是\cite{24_DBLP:conf/iccv/YangGWHYL19}所提方法。理论上,只有实体短语可以从图片中检测到,没有在图片中检测到对应视觉目标的名词短语不被列为短语实体。另一种得到对齐的名词短语与视觉目标的方法是使用已标注的数据,Flickr30K Entities\cite{29_DBLP:conf/iccv/PlummerWCCHL15,30_DBLP:journals/ijcv/PlummerWCCHL17}就是标注了图片与文本中实体对应关系的数据集。该数据集与Multi30K高度重合,因此可以直接利用到本章所提方法上。对于从图片中截取到的视觉目标图片,本章利用在ImageNet\cite{31_DBLP:journals/ijcv/RussakovskyDSKS15}上预训练的ResNet-50\cite{32_DBLP:conf/cvpr/HeZRS16}提取2048维的全局特征。

\subsection{基于RNN的模型设置}
\label{sec:3_rnn_setup}
对于基于RNN的翻译模型,本章采用的是带有注意力机制并基于编码器-解码器结构的神经翻译模型\cite{4_luong-etal-2015-effective}。其中,编码器是一个维度维500的单层双向的LSTM,重构模型和翻译模型的解码器都是单层500维的LSTM,词嵌入层的维度同样设置为500。编码器和解码器中的dropout均设置为0.3。模型参数是从$(-0.1,+0.1)$区间的均匀分布中采样初始化的,其中偏置项(bias)初始化为0。训练时采用Adam\cite{34_DBLP:journals/corr/KingmaB14}优化器优化模型参数,学习率设为定值0.002。批数据大小为40。最终模型是根据验证集在BLEU4\cite{42_papineni-etal-2002-bleu}上的表现进行选择。在训练过程中,当模型在验证集上的BLEU4值超过10个迭代轮次不再提升,则停止训练并作为最终用于评测的模型。

\subsection{基于Transformer的模型设置}
\label{sec:3_transformer_setup}
对于基于Transformer的翻译模型,本章将词向量的维度设置为128,隐层状态维度设置为256,自注意力的头数为4,模型编码器和解码器的层数均为4,dropout设置为0.2。区别于基于RNN的模型,Transformer采用的是合并的词表,因此词表大小始终设置为合并词表的大小27226。模型优化采用的是Adam优化器,其中$\beta_1=0.9$,$\beta_2=0.998$,$\epsilon=10^{-9}$。本章与文献\cite{5_DBLP:journals/corr/VaswaniSPUJGKP17}相同采用预热和衰减策略来提高学习率,预热步骤为4000,总训练步骤为80000。训练目标中设置平滑标签(label smoothing)$\epsilon_{ls}=0.1$。以上模型参数与文献\cite{33_yin-etal-2020-novel}。

\subsection{多任务训练}
本章所提方法需要对重建任务和翻译任务采用多任务训练的方式进行模型优化。多任务训练采用的是两个任务随机交替训练的方式,并利用\ref{sec:3_multitask}节中所介绍的超参数$\omega$控制翻译任务的训练概率。因为两个任务所使用数据的比例相同,因此本章设置$\omega=0.5$。

\subsection{对比模型}
基于RNN的模型:
\begin{itemize}
    \item \textbf{NMT:}基于LSTM的纯文本神经翻译模型,其模型配置与\ref{sec:3_rnn_setup}节中的描述保持一致。
    \item \textbf{pRCNNs\cite{35_huang-etal-2016-attention}:}该模型在编码阶段将每个视觉目标特征与源语言句子编码一次,在解码阶段,解码器选择关注与当前解码步骤相关的视觉目标所对应的文本序列。
    \item \textbf{DATT\cite{36_calixto-etal-2017-doubly}:}该模型设置了两个注意力机制模块,其中一个用于关注文本信息,另一个用于关注图片的栅格特征。并设置了一个门控值控制图片信息输入的量。
    \item \textbf{Imagination\cite{37_elliott-kadar-2017-imagination}:}该方法将源语言句子编码后,利用编码后的隐层表示预测输入句子所对应的图片。该方法同样使用了多任务的方式。
    \item \textbf{VMMT\cite{38_calixto-etal-2019-latent}:}该方法中的$ \mathrm{VMMT_C} $与$ \mathrm{VMMT_F} $采用隐变量建模语义信息,在翻译过程中利用视觉信息与文本信息使变分编码器融合跨模态语义。
\end{itemize}

基于Transformer的模型:
\begin{itemize}
    \item \textbf{Transformer:}基于Transformer的纯文本神经翻译模型,其模型配置与\ref{sec:3_transformer_setup}节保持一致。
    \item \textbf{DelMMT\cite{39_ive-etal-2019-distilling}:}该模型提出使用推敲做多模态二次接码。在二次解码中融合源语言、目标语言以及视觉信息。
    \item \textbf{MMT-TF\cite{40_yao-wan-2020-multimodal}:}该工作设计了一种多模态注意力模块,该模块需要链接源语言句子的表示和图像特征作为自注意力模块的查询。
    \item \textbf{GAMMT\cite{41_DBLP:journals/corr/abs-2103-08862}:}使用Gumbel-Sigmoid改造注意力机制,帮助翻译模型关注到图片中与文本内容更相关的区域。
    \item \textbf{GMMT\cite{33_yin-etal-2020-novel}:}该模型视源语言句子与图片中的视觉目标为一个多模态图结构,然后利用设计的基于图的跨模态编码器进行编码,最终解码出目标端句子。
\end{itemize}


\section{实验结果}

\section{本章小结}
