\section{相关工作}
%本章工作主要针对将加入对抗样本的对比学习方法应用到融合图片信息的神经机器翻译中,因此相关工作可以分为以下三个部分:
本章工作主要采用在对比学习方法中加入对抗样本的方式,迫使模型将图片信息融合到文本表示中。因此,相关工作可以分为以下三个部分:

%传统方法:都有哪些方法,这些方法出现的问题,
{\sffamily (1)图片信息辅助式神经机器翻译}
%这里指出简单的方法,和复杂的方法
在融合图片信息的神经机器翻译方法中,将图片作为翻译模型输入的一部分,并将其用于辅助源语言文本的编码,或在目标语言生成过程中提供外源参考信息的图片辅助式神经机器翻译是目前最常见的一类。图片可以以多种方式与神经翻译任务相结合,而为了更有效地利用图片信息,研究者们设计了从简单到复杂的图片信息融合方式。\tcite{elliott2015multilanguage}将图片的全局特征输入到基于循环神经网络的翻译模型中,将其与源语言句子编码后的隐层向量连接后用于初始化解码器。\tcite{calixto2017incorporating}则尝试了更多图片全局特征的使用方法,例如直接初始化编码器,直接初始化解码器以及当做词串接到输入句子的词向量后。或者可以利用图片栅格特征可以当做图片特征序列的特性,采用注意力机制为解码器提供动态的上下文信息\pcite{calixto2017doubly,libovicky2018input}。这些方法均是通过简单的模型改动达到使神经机器翻译模型支持视觉特征输入的目的。然而\tcite{elliott2018adversarial}的对抗评估结果表示以简单的方式将图片输入给模型并不能使图片信息得到很好的利用。在后续的研究进程中涌现了更多复杂的图片信息融合方法。\tcite{ive2019distilling}将图片信息作用到解码阶段后的推敲网络(deliberation network)进行二次解码。\tcite{yin2020novel}将句子与图片中的视觉目标视为图的关系,采用基于图的编码器融合跨模态信息。
然而\tcite{caglayan2019probing}的实验表明,在文本信息缺失的情况下,即便采用简单的图片输入方式,翻译模型也会更多地利用图片信息。因此,对于图片辅助式神经机器翻译方法,需要更多地关注如何使翻译模型更多地关注图片中的信息,从而加强图片信息的辅助作用。

%对比学习方法:主要有NMT的方法,MMT也有相关的概念,但与本文的不同
{\sffamily (2)基于对比学习的神经机器翻译}

对比学习方法已经广泛应用于计算机视觉和自然语言处理的相关研究中。该方法的作用是将样本空间中具有相似语义信息的样本点拉近,并增加不相关样本之间的表示距离。通常,相关样本之间可互称为正样本,不相关样本之间可互称为负样本。
\tcite{pan2021contrastive}在一个多语言神经机器翻译系统中加入了对比学习方法。在训练期间,每个双语翻译对都是正样本,为每对正样本从数据集中随机选取不相关句子作为负样本。在对比损失函数的配合下,针对多语言翻译的交叉熵损失函数能够帮助模型学习到多语言的统一表示空间,从而达到提升多语言之间翻译性能并帮助零资源翻译的目的。
\tcite{li2021unimo}提出了一种应用于多模态理解与生成任务的多模态预训练框架。该工作通过对比学习方法建立文本与图片之间的跨模态统一表示空间。在构建过程中,通过计算文本表示与图片表示之间的相似度,实现对来自不同模态信息的语义距离的计算。这种方式能够帮助模型获得更多的跨模态信息,从而更好地服务于下游任务。
\tcite{elliott2017imagination}在训练纯文本的神经机器翻译模型中,引入了“想象力”机制作为翻译的子任务。该机制将文本表示映射到视觉特征的表示空间,尝试通过“想象”的方式使文本表示尽可能地接近图片中的语义。该过程就是利用了一种对比损失函数,使文本表示尽可能接近对应图片的特征表示,并远离同批数据下其它图片的特征表示。
\tcite{kiros2014unifying}则是在融合图片信息的神经机器翻译中,通过建立句子与图片统一的表示空间达到应用图片信息提升文本表示的能力的目的,而在构建中同样采用了对比学习的方法。该工作在对比学习中同样加入了对抗样本作为负样本的方式,尝试拉近源语言句子与相关图片在表示空间的距离,并尝试推开与句子不相关的图片,以及推开与图片不相关句子。与该工作不同的是,本章所提方法虽然也引入了对抗样本作为负样本,但并不是针对图片与文本之间语义表示的对比学习,而是将图片信息融合到文本的表示中,针对不同文本表示的对比学习,并在该过程中加入了译文作为锚点,使对比学习作用到翻译而不是统一跨模态语义空间。

%对抗训练方法:主要用于测试
{\sffamily (3)对抗训练方法}

对抗训练方法常用于在训练神经机器翻译模型时,通过引入对抗样本的方式提升系统的鲁棒性。然而,在融合图片信息的神经机器翻译中,对抗样本更多用于测试输入的图片能否在翻译的过程中起作用。
\tcite{elliott2018adversarial}提出采用对抗评估测试翻译模型是否利用到视觉信息。该研究在融合图片信息的神经机器翻译模型中输入与文本相对应和不对应的图片,然后通过测试模型的翻译准确率变化判断模型在翻译过程中是否受到输入图片的影响。
\tcite{wu2021good}和\tcite{li2021vision}在分析视觉信息是否在翻译模型中起作用时,同样使用了对抗样本的方法,并认为图片与噪音的作用相似,是通过正则化(regularization)的方式提升了翻译模型的性能。
\tcite{elliott2018adversarial}没有采用图片的对抗样本,而是采用了文本对抗输入的方式测试融合图片信息的神经机器翻译模型能否通过图片中的信息纠正对抗文本中的错误。其结果表明图片信息无法做到纠正文本错误。这一结果也说明了虽然模型的输入同时包含了来自图片和文本两个模态的信息,但融合图片信息的神经机器翻译仍然是文本信息为主图片信息为辅的。
与以上工作在测试阶段应用对抗样本不同的是,本章所提方法将对抗样本应用于对比学习的训练过程。对抗样本作为负样本能够在对比学习的语义聚类过程中,帮助模型结合更细粒度的语义信息,从而使模型能够分辨出输入的图片信息是否与文本内容一致。