\chapter{基于图文对比对抗训练的神经机器翻译}
% 本章要强调 隐式方法在理论上对视觉信息的利用率最大,能够最大程度提升翻译,但实际研究中却很难实现
% 已有工作
前两章我们聚焦于图片信息增强式的神经翻译模型,通过在模型训练阶段引入视觉信息增强模型的表示能力,从而提升纯文本翻译模型的翻译质量。
% 存在的问题
然而这种做法规避了图片信息在生成翻译过程中的辅助作用,使得视觉信息的作用没有得到充分的发挥。
% 本章所针对的问题
为此本章将主要探索图片信息辅助式的神经机器翻译方法,尝试将图片信息融入句子级别的语义信息中。在针对已有方法的讨论中我们知道,此类方法属于隐式跨模态信息融合方法,需要解决神经翻译模型对视觉信息不敏感的问题。
% 具体做法
因此,本章提出一种图文对比对抗训练方法,尝试提升视觉信息在语义表示学习过程中的参与度。在模型输入为文本加图片的情况下,通过与我们设计的图文对抗样本进行对比,使模型重视图片输入信息与文本描述之间的差异,从而将图片信息融合到文本的表示中。
% 结果
在英译德、英译日和英译印三个翻译任务上的实验表明,本章所提方法能够有效提升翻译准确率的同时,增强了模型对视觉信息的敏感度。

\section{引言}
% 背景:
得益于神经网络方法的快速发展,自然语言处理和计算机视觉等领域均得到了飞速的进步,同时也为跨领域的信息融合带来了可能。在神经机器翻译中融合图片信息就是其中一种跨模态信息融合方法。跨模态的信息融合是为了解决一些在翻译中难以解决的问题,例如文本中存在歧义,表达不完整,文本表示欠拟合等问题。图片中往往包含着更丰富的语义信息。因此,如何将图片信息融合到翻译中成为了研究者们所关注的问题,相关方法也层出不穷。

% 相关研究:
目前已有的研究主要关注如何将整个图片中的信息融合到神经机器翻译模型的翻译过程中。由于图像分类任务所使用的卷积神经网络,如Resnet-50\cite{32_DBLP:conf/cvpr/HeZRS16},取得了非常好的效果,因此多数研究这尝试将预训练的卷积神经网络与神经翻译模型相结合。例如部分研究将图片经过卷积神经网络提取得到全局特征后,将其作为一个包含完整的上下文信息的特征向量以各种方式输入到翻译模型中,从而完善翻译模型在编码过程所获得的语义表征或为解码过程提供外源信息以供参考\cite{52_DBLP:journals/corr/ElliottFH15,18_DBLP:conf/emnlp/CalixtoL17,22_li-etal-2021-vision,20_wu-etal-2021-good}。
文献\cite{36_calixto-etal-2017-doubly,47_DBLP:conf/wmt/LibovickyHM18}尝试利用图片局部特征与注意力机制配合获图片的局部动态上下文,以达到在解码过程中融合与当前解码步骤最相关的视觉信息。此类方法中,图片作为一个句子级别的语义信息提供者作用到翻译过程中,试图使翻译模型最大化地利用图片中所包含的信息。
%因视觉信息与文本信息的融合与利用为通过模型学习的方式,而难以明确图片信息在翻译中的作用过程,本文称此类方法为隐式跨模态信息融合方法。

% 存在的问题:
然而,这些方法均以隐式的方式将图片信息与文本信息相结合,这些融合图片信息的神经机器翻译模型是如何应用视觉信息以及为何输入图片能够使得翻译性能提升都是不明确的,本文中,我们称此类方法为隐式跨模态信息融合方法。更重要的是,部分方法存在翻译质量提升不稳定的情况,不同的模型实现方案或不同的模型初始化条件下模型的翻译性能表现差异较大。尽管在句子级语义单位可以最大化地利用图片中所携带的视觉信息,但神经翻译模型似乎不能很顺利地将其利用。

% 本章提出方法
为此,本章提出了一种明确的图片信息融合方法。文献\cite{53_caglayan-etal-2019-probing}发现,当文本中部分内容或信息缺失,尤其是实体词缺失时,神经翻译模型对视觉信息最敏感。为了将图片中的视觉信息明确并有效地融合到模型中,本文以实体为语义单元融合图片信息,其中实体以短语实体或名词实体两种粒度呈现。区别于文献\cite{53_caglayan-etal-2019-probing}通过创造的文本内容缺失环境使隐式跨模态信息融合有效,本文采用显式跨模态信息融合方法,直接利用视觉特征替换实体作为模型输入,规避了神经翻译模型对视觉信息不敏感的问题。为了实现跨模态的信息融合,本文设计了一个重构模型来实现多模态输入到纯文本输出的任务。该模型以图片的视觉目标联合实体被替换的源语言文本为输入,以原文或译文为输出,通过端到端的学习将视觉目标中的视觉信息融合到实体的表示中。最后,本文利用了多任务学习方法,将跨模态文本重构任务与翻译任务相结合。通过多种参数共享机制,使翻译模型能够充分利用到重构任务重所学习到的视觉信息,从而提升翻译质量。

本章主要贡献如下:

(1)本章提出了一种基于跨模态文本重构的神经机器翻译方法(neural machine translation based on cross-modal text reconstruction, CTR-NMT), 并且将所提方法应用到了RNMT和Transformer两种框架。本章详细对比了多种参数共享方案,在推断(inference)时没有额外的图片输入的情况下,达到了与其它模型可比的翻译准确率。与纯文本基线模型相比,有着稳定的翻译质量的提升。

(2)本章提出了一种明确的融合图片信息的方法,即显式跨模态信息融合法。将图片中的视觉目标信息明确的作用到在文本中相对应的实体上,并在基于短语和基于词两种粒度的跨模态融合方案进行了实验对比,发现相比于短语实体,在词实体上融合图片信息效果更佳。

(3)本章对视觉目标的作用对象进行了分析,并发现本章所提的明确的图片信息作用方法,使名词实体的翻译准确率得到了更进一步的提升,从而提升了神经机器翻译模型的翻译质量。

\section{相关工作}
本章工作主要针对将加入对抗样本的对比学习方法应用到融合图片信息的神经机器翻译中,因此相关工作可以分为以下三个部分:

%传统方法:都有哪些方法,这些方法出现的问题,
{\sffamily (1)图片信息辅助式神经机器翻译}
%这里指出简单的方法,和复杂的方法
在融合图片信息的神经机器翻译方法中,将图片作为翻译模型输入的一部分,并将其用于帮助源语言文本的编码或在目标语言生成过程中提供外源参考信息的图片辅助式神经机器翻译是目前最常见的一类。图片可以以多种方式与神经翻译任务相结合,而为了更有效地利用图片信息,研究者们设计了从简单到复杂的图片信息融合方式。文献\cite{52_DBLP:journals/corr/ElliottFH15}将图片的全局特征输入到基于循环神经网络的翻译模型中,将其与源语言句子编码后的隐层向量连接后用于初始化解码器。文献\cite{18_DBLP:conf/emnlp/CalixtoL17}则尝试了更多图片全局特征的使用方法,例如直接初始化编码器,直接初始化解码器以及当做词串接到输入句子的词嵌入后。文献\cite{36_calixto-etal-2017-doubly,47_DBLP:conf/wmt/LibovickyHM18}则利用图片栅格特征可以看作图片特征序列的特性,采用注意力机制为解码器提供动态的上下文信息。这些方法均是通过简单的模型改动达到使神经机器翻译模型支持对视觉特征输入的目的。然而文献\cite{23_elliott-2018-adversarial}的对抗评估则表示以简单的方式将图片输入给模型并不能使图片信息得到很好的利用。在后续的研究进程中涌现了更多复杂的图片信息融合方法。文献\cite{39_ive-etal-2019-distilling}将图片信息作用到解码阶段后的推敲网络(deliberation network)进行二次解码。文献\cite{33_yin-etal-2020-novel}将句子与图片中的视觉目标视为图的关系,采用基于图的编码器融合跨模态信息。
然而文献\cite{53_caglayan-etal-2019-probing}的实验表明,在文本信息缺失的情况下,即便采用简单的图片输入方式,翻译模型也会更多地利用图片信息。因此,对于图片辅助式神经机器翻译方法,需要更多地关注如何使翻译模型更多地关注图片中的信息,从而加强图片信息的辅助作用。

%对比学习方法:主要有NMT的方法,MMT也有相关的概念,但与本文的不同
{\sffamily (2)基于对比学习的神经机器翻译}

对比学习方法已经广泛应用于计算机视觉和自然语言处理的相关研究中。该方法的作用是将样本空间中具有相似语义信息样本点拉近,并增加不相关样本之间的间隙。通常,相似样本之间可互称为正样本,不相关样本之间可互称为负样本。
文献【】在一个多语言神经机器翻译系统中加入了对比学习方法。在训练期间,每个双语翻译对都是正样本,为每对正样本从数据集中随机选取不相关句子作为负样本。在对比损失函数的配合下,针对多语言翻译的交叉熵损失函数能够帮助模型学习到多语言的统一表示空间,从而达到提升多语言之间翻译性能并帮助零资源翻译的目的。
文献【】提出了一种应用于多模态理解与生成任务的多模态预训练框架。该工作通过对比学习方法建立文本与图片之间的跨模态统一表示空间。在构建过程中,通过计算文本表示与图片表示之间的相似度,实现对来自不同模态信息的语义距离的计算。这种方式能够帮助模型获得更多的跨模态信息,从而更好地服务于下游任务。
文献\cite{37_elliott-kadar-2017-imagination}在训练纯文本的神经机器翻译模型中,引入了“想象力”机制作为翻译的子任务。该机制将文本表示映射到视觉特征的表示空间,尝试通过“想象”的方式使文本表示尽可能地接近图片中的语义。该过程就是利用了对比损失函数,使文本表示尽可能接近对应图片的特征表示,并远离同批数据下其它图片的特征表示。
文献【】则是在融合图片信息的神经机器翻译中,通过建立句子与图片统一的表示空间达到应用图片信息提升文本表示的能力的目的,而在构建中同样采用了对比学习的方法。该工作在对比学习中同样加入了对抗样本作为负样本的方式,尝试拉近源语言句子与相关图片在表示空间的距离,并尝试推开与句子不相关的图片,以及推开与图片不相关句子。与该工作不同的是,本章的方法虽然也引入了对抗样本作为负样本,但并不是针对图片与文本之间语义表示的对比学习,而是将图片信息融合到文本的表示中,针对不同文本表示的对比学习,并在该过程中加入了译文作为锚点,使对比学习作用到翻译而不是跨模态表示。

%对抗训练方法:主要用于测试
{\sffamily (3)对抗训练方法}

对抗训练方法常用于在训练神经机器翻译模型时,通过引入对抗样本的方式提升系统的鲁棒性。然而,在融合图片信息的神经机器翻译中,对抗样本更多用于测试输入的图片能否在翻译的过程中起作用。
文献\cite{23_elliott-2018-adversarial}提出采用对抗评估来测试翻译模型是否利用到视觉信息。该研究在融合图片信息的神经机器翻译模型中输入与文本相对应和不对应的图片,然后通过测试模型的翻译准确率的变化来判断模型在翻译过程中是否受到输入图片信息的影响。
文献\cite{20_wu-etal-2021-good,22_li-etal-2021-vision}在分析视觉信息是否在翻译模型中起作用时,同样使用了对抗样本的方法。该工作认为图片与噪音的作用相似,是通过正则化(regularization)的方式提升了翻译模型的性能。
文献【】则采用了文本对抗输入的方式测试融合图片信息的神经机器翻译模型能否通过图片中的信息纠正对抗文本带来的错误。其结果表明图片信息无法做到纠正文本错误。这一结果也说明了虽然模型的输入同时包含了来自图片和文本两个模态的信息,但融合图片信息的神经机器翻译仍然是文本信息为主图片信息为辅的。
与以上工作在测试阶段应用对抗样本不同的是,本章所提方法将对抗样本应用于对比学习的训练过程。对抗样本作为负样本能够在对比学习的语义聚类过程中,帮助模型结合更细粒度的语义信息,从而使模型能够分辨出输入的图片信息是否与文本内容一致。
\section{方法描述}
\label{sec:5_method}

对于一个标准的图片信息辅助式神经机器翻译,其输入为一个源语言句子$X$和一个与其对应的图片$I$,然后利用这两个模态的信息生成翻译$Y$。然而,现有的很多方法对输入图片$I$中的信息并不敏感。这是因为翻译任务所使用的语料中,大部分$X$与$Y$之间有着很好的语义对齐关系。这使得翻译模型在不依靠图片输入的情况下,就可以在训练阶段学习到一组相对不错的模型参数,使模型很容易收敛到一个不需要图片信息的局部最优解。为了使模型充分利用为待翻译句子所提供的视觉信息,本章提出了基于图文对比对抗训练的神经机器翻译方法。本节将首先介绍本章所使用的翻译模型结构,然后详细介绍如何通过对比对抗训练增加模型对视觉信息的敏感度。

\subsection{模型结构}
\label{sec:5_architecture}
本章采用了一种句子级跨模态语义融合模型结构,其输入为一个文本序列,和一个可选的视觉输入,该视觉输入为文本序列所对应的完整图片以及图片内部的数个视觉目标,如图【】所示。该模型的编码器和解码器为一般的Transformer结构。其中,Transformer的编码器负责跨模态信息融合。为了使Transformer支持文本加图片形式的输入,我们采用了一个多模态嵌入层。该嵌入层共分为4个子层:词嵌入层,视觉特征层,模态分割层,位置编码层。

(1){\sffamily 词嵌入层:}为了支持图片的输入,需要在词嵌入层所对应的词表中,加入一些特殊字符:“<seg>”将词嵌入层输入的前后两部分分割为文本和视觉两个序列;“<img>”代表着放置完整图片的位置,每个输入序列仅一个;“<bbx>”代表放置图片中的视觉目标的位置,每个输入序列可以有多个;“<end>”代表着多模态输入序列的结尾。


(2){\sffamily 视觉特征层:}该层每个位置的输入与词嵌入层的特殊字符相对应,其中“<img>”的对应位置放置输入的完整图片的全局特征,“<bbx>”对应位置放置视觉目标的图片全局特征,其它位置则输入零向量。


(3){\sffamily 分割层:}这一层的作用相对简单,主要用作标识每个位置的输入属于文本序列还是图片序列。因此一共只有“文本模态”和“视觉模态”两个值,每个在模型中是一个与词嵌入层维度相同的向量。其中“<seg>”之前的文本序列均对应着“文本模态”,“<seg>”以及其后面的序列均为“视觉模态”。


(4){\sffamily 位置向量层:}这一层与一般的位置向量作用相似,能够表示输入序列中的绝对位置关系。区别在于,当输入的视觉目标与文本中的某些词有对应关系时,可以将视觉目标的位置设置为与其对应的词相同的位置,从而达到加强图片信息作用准确性的目的。如图【】所示,图片输入序列中的“帽子”的绝对位置为5,与文本序列中的“hat”保持一致。

在解码阶段,解码器仅接收“文本模态”的编码结果作为输入。这是因为本章的CAT方法旨在帮助神经机器翻译模型将视觉信息融合到文本表示中。而该过程发生在编码阶段,即没有针对解码器采取翻译以外的优化方法。因此,基于一般的翻译模型对视觉信息不敏感的假设,其内部模块,如解码器,往往也会忽略视觉信息的输入。所以,将“视觉模态”部分的隐层单元传递给解码器是多余的。

图【】中可以看到,CAT-MMT的目标函数包含了两个部分:融合图片信息的神经机器翻译所需的交叉熵损失函数和CAT所需的对比损失函数。其中交叉熵损失函数为:
\begin{equation}
    L_{CE}(\phi, \theta, \psi)=-\sum_j^M \log p(y_j|y_{<j},X,I)
\label{eq:5_cross_entropy}
\end{equation}
其中,$y_j \in Y,j=1,\cdots,M$,$\phi$为解码器参数,$\theta$为编码器参数,$\psi$为多模态嵌入层的参数。该损失函数与一般的融合图片信息的神经机器翻译模型的相似,均是通过源语言文本$X$和对应图片$I$为输入信息,生成翻译$Y$的形式。在我们的方案中,$I$代表了完整的图片与其中的多个视觉目标的组合。该损失函数无法反映出的信息是,本章的跨模态信息融合过程,仅发生在编码端。图【】中除了平均池化(average pooling)层,其它参数均需要通过翻译任务来优化。

\subsection{对比对抗训练}
\label{sec:5_cat}
正如\ref{sec:5_architecture}小节所介绍,本章是在Transformer的编码器中实现将跨模态信息融合到文本的表示中。然而,基于一般的翻译模型对视觉信息不敏感的假设,在没有额外的引导下,编码器同样会忽略图片信息的存在。为了有效地融合视觉信息,本章提出了图文对比对抗方法来实现在编码过程将图片中的视觉信息融合到文本的表示中。CAT方法主要包含三部分:对比学习,对抗样本以及双向翻译训练。

{\sffamily 对比学习}

对比学习方法能够在语义表示空间中拉近相似的样本,推离不相关的样本。针对融合图片信息神经翻译中,将目标译文$Y$作为锚点,将编码端的输入$X+I$作为正样本。在$X$与$Y$的双语统一表示空间中,对比学习方法能够拉近$X+I$与$Y$之间的距离,并将其它的所有样本视为负样本,拉开与$X+I$和$Y$之间的距离。图【】中每个圆代表文本语义表示空间的一个点,其中图【】(a)和图【】(b)展示了对比学习将表示空间的样本分簇归类的能力,图【】(b)中的每个簇可视为一个双语平行句对。常规的对比学习损失函数为:
\begin{equation}
    \mathcal{L}_{CL}(\theta, \psi)=-\log\ \frac{e^{\mathrm{sim}(\mathcal{R}(Y),\mathcal{R}(X+V))/\tau}}{\sum_{Z\in N}e^{\mathrm{sim}(\mathcal{R}(Y),\mathcal{R}(Z))/\tau}}
    \label{eq:5_contrastive_learning}
\end{equation}

其中$\mathrm{sim}(\cdot,\cdot)$为余弦相似度函数,用于计算两个表示向量的语义相似度;$\tau$为温度,控制区分正负样本的能力;$\mathcal{R}(\cdot)$代表平均池化,正如图【】所示,对比损失的输入是文本表示经平均池化后的结果;$N$代表着负样本集。

$\mathcal{L}_{CL}$与$\mathcal{L}_{CE}$之间共享编码器和嵌入层的参数,因此本章所采用的对比学习只针对编码器进行参数优化。
\section{实验设置}
\label{sec:5_setup}

\subsection{实验数据}
\label{sec:5_datasets}

\begin{table}[!htbp]
    \bicaption{数据集情况统计结果}{Statistic results about datasets}
    \label{tab:5_datasets}
    \centering
    \footnotesize% fontsize
    \setlength{\tabcolsep}{4pt}% column separation
    \renewcommand{\arraystretch}{1.2}%row space 
    \begin{tabular}{cccccc}
    \hline
    数据集 & 翻译任务 & 训练集 & 验证集 & 测试集 & 平行句对数量 \\\hline
    Multi30K\pcite{elliott2016multi30k}          & EN$\rightarrow$DE   & 29,000 & 1,014 & 1,000/1,000/461 & 32,475  \\
    Flickr30KEnt-JP\pcite{nakayama2020flickr30kentjp}   & EN$\rightarrow$JP   & 29,783 & 1,000 & 1,000           & 158,915 \\
    HVG\pcite{parida2019hindi}               & EN$\rightarrow$HI   & 28,932 & 998 & 1,595/1,400     & 32,925  \\
%    CLT-MMT bi $\cup B_{\setminus X}$ & 38.5  & 57.5  & 31.0  & 51.9  & 27.5  & 47.4  \\
%    CAT-MMT bi $\cup B_{\setminus X}$ & 38.5  & 57.5  & 31.0  & 51.9  & 27.5  & 47.4  \\
     \hline
    \end{tabular}%
\end{table}%

为了验证方法的有效性,本章分别在三个语言对上进行了测试,包括英德翻译(EN-DE)、英日翻译(EN-JP)和英印翻译(EN-HI)。这些数据集的数据统计情况如表\ref{tab:5_datasets}所示,其详细介绍如下:

(1){\sffamily 英德翻译:}Multi30K是融合图片信息的神经机器翻译任务普遍采用的数据集。该数据集中每张图片对应一句英文描述和一个德文翻译,并划分为训练集、验证集和测试集三部分,其中测试集又分为Test2016、Test2017和ambiguous MSCOCO 2017三个测试集。ambiguous MSCOCO 2017是专门针对歧义词问题设计的测试集。\ref{sec:5_architecture}小节中提到,可以为视觉目标与文本中的对应词提供相同的位置向量。针对Multi30K数据集,本章采用Flickr30K Entities提供的视觉目标与文本短语的实体对齐关系,文本中的对应词仅选择对应短语中的名词。由于Flickr30K Entities没有为Test2017和ambiguous MSCOCO 2017两个测试集提供实体对齐关系,因此我们采用\ref{sec:3_entity_extraction}节应用的视觉目标提取工具获得该对齐关系。本章采用Moses SMT\cite{44_koehn-etal-2007-moses}工具包对数据进行分词(tokenization)和归一化(normalization)处理。为了防止对视觉目标与名词对应关系的破坏,本章并没有采用双节编码技术\cite{27_sennrich-etal-2016-neural}(byte pair encoding, BPE)或WordPiece\cite{28_DBLP:journals/corr/WuSCLNMKCGMKSJL16}进行分词操作。

(2){\sffamily 英日翻译:}Flickr30KEnt-JP与Multi30K都是源自Flickr30K数据集。因此两者所使用的图片是一样的,区别在于Flickr30KEnt-JP是将Flickr30K的5个英文描述都翻译到了日文。所以该数据集中每张图片对应了5个英文描述和5个对应的日文翻译。相应地,也对该数据集的训练集、验证集和测试集的数据划分做了改变。

(3){\sffamily 英印翻译:}英语到印地语的翻译任务采用了HVG(Hindi Visual Genome)数据集。该数据集的图片来源于Visual Genome数据集。区别于Multi30K和Flickr30KEnt-JP,HVG中的每段英文描述是针对图片中的一个区域,这意味着其英文描述相对更短更简单。另外,该数据集提供了每段描述在图片中对应的区域。在实验过程中个,我们将完整的图片输入到“<img>”的对应位置,将与描述相关的图片区域输入到“<bbx>”对应的位置。

\subsection{模型设置}
\label{sec:5_model_setup}

本章所提方法在基于Transformer的翻译模型基础上进行的实验。因模型参数规模受数据集的大小的限制,我们选择了小规模的参数设置。Transformer的词向量维度为128,隐层向量维度为256。编码器和解码器的层数均为4。多头注意力机制的头数为4。在训练过程中dropout设置为0.2。批数据大小设置为源语言以及目标语言最多不超过2000个单词。本章采用Adam优化器在模型的训练过程中进行参数优化其中$\beta_1=0.9$,$\beta_2=0.998$,$\epsilon=10^{-9}$。本章与文献\cite{5_DBLP:journals/corr/VaswaniSPUJGKP17}相同采用预热和衰减策略来提高学习率,预热步骤为4000,总训练步骤为80000,取训练完成后的模型参数用于测试。训练目标中设置平滑标签$\epsilon_{ls}=0.1$。以上模型参数与文献\cite{33_yin-etal-2020-novel}。

公式\ref{eq:5_contrastive_learning}中的$\tau$设置为0.1,公式\ref{eq:5_combine_with_nmt}中的超参数$\lambda$设置为0.3。在评估英德翻译的结果时,采用BLEU4\cite{42_papineni-etal-2002-bleu}和METEOR\cite{46_denkowski-lavie-2014-meteor}两个评估指标。英日翻译和英印翻译仅采用BLEU4作为评估指标。

\subsection{对比模型}

本章所选择的对比模型可分为两类:图片辅助式神经翻译模型和图片增强式神经翻译模型。图片辅助式神经翻译模型在生成翻译的过程中会以源语言句子和对应图片作为参考生成译文,图片主要起到辅助翻译过程的作用。图片增强式神经翻译模型则是利用图片中的视觉信息增强模型的表示能力,在测试阶段一般不需要图片输入。本章所选择的对比模型均是以Transformer作为基础模型结构:
\begin{itemize}
    \item \textbf{SerialAtt\cite{47_DBLP:conf/wmt/LibovickyHM18}:}该模型在解码过程中采用了多个交叉注意力串联的形式,每个交叉注意力模块对应了不同的信息来源。在融合图片信息的模型中,两个交叉注意力模块分别用于采集源语言和图片中的信息。
    \item \textbf{MMT-TF\cite{40_yao-wan-2020-multimodal}:}该工作设计了一种多模态注意力模块,该模块需要链接源语言句子的表示和图像特征作为自注意力模块的查询。
    \item \textbf{OVC\cite{48_DBLP:conf/aaai/WangX21}:}该方法设计的模型以视觉目标作为图片输入,并设计了以针对视觉目标的损失函数,通过对输入视觉目标进行掩码操作使模型对具有相关信息的图片视觉目标敏感,并忽略那些不相关的视觉目标。
    \item \textbf{GAMMT\cite{41_DBLP:journals/corr/abs-2103-08862}:}使用Gumbel-Sigmoid改造注意力机制,帮助翻译模型关注到图片中与文本内容更相关的区域。
    \item \textbf{GMMT\cite{33_yin-etal-2020-novel}:}该模型视源语言句子与图片中的视觉目标为一个多模态图结构,然后利用设计的基于图的跨模态编码器进行编码,最终解码出目标端句子。
    \item \textbf{CTR-NMT:}该模型为本文第3章所设计的方法,采用了词级实体替换方案利用独享解码器重构源语言,是一种图片增强式神经翻译模型。
    \item \textbf{CER-NMT:}该模型为本文第4章所设计的方法,采用了双向实体重构方法,是一种图片增强式神经翻译模型。
    \item \textbf{Transformer:}基于Transformer的单向纯文本神经翻译模型,其模型配置与\ref{sec:5_model_setup}节保持一致。
    \item \textbf{MM-Transformer:}该模型为本章所设计的不使用对比损失的神经机器翻译模型,该模型与一般的融合图片信息的翻译模型相似,为一个单向翻译模型。
    
\end{itemize}

\section{实验结果}


\section{本章小结}
为了解决图片辅助式的神经机器翻译对视觉信息不敏感的问题,本章提出了一种在对比学习中加入对抗样本的方法。在双向翻译训练方法的配合下,翻译模型以粗粒度的方式将不同的翻译句对聚类,以细粒度的方式区分翻译句对和对抗样本,从而实现了将输入图片中的信息融合到文本表示中。
本章在三个数据集进行了实验。实验结果表明,采用对比对抗训练的神经翻译模型在提升翻译准确率的同时,对图片信息更敏感。