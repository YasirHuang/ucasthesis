\chapter{结束语与展望}

\section{工作总结}
%研究背景
近年来,深度学习技术的研究与应用得到了快速的发展,人工智能的影响也逐渐深入到人们工作与生活的方方面面。因此,机器翻译的相关研究与应用也不再局限于面向单个句子或文本的翻译任务。在翻译任务中加入视觉信息,是靠近人类使用自然语言习惯的重要一步。
%研究意义
融合图片信息的神经机器翻译通过在模型的训练、测试或者应用中融入图片信息,强化了翻译模型的表征能力或削弱文本中常见的病句问题对翻译带来的影响。因此,跨模态信息融合方法在学术与工业领域都得到了广泛的关注。
%存在的问题
在神经机器翻译中融合图片信息,一方面需要关注如何使用图片信息能够提升翻译的质量,另一方面需要注意噪音的正则化效果带来的翻译准确率的提升。
%本文解决的问题
本文分别使用显式与隐式两种图片信息作用方式,在基于图片信息增强的和基于图片信息辅助的两类跨模态信息融合方法下,探究在神经机器翻译中融合图片信息的有效方法,提高译文的翻译质量。
本文研究内容总结如下:

{\sffamily 1. 基于跨模态文本重构的神经机器翻译方法}

% 面对什么问题
在神经机器翻译中融合图片信息具有直接从数据中学习并融合跨模态信息的优点,从而达到充分利用视觉信息提升翻译质量的目的,但因难以明确图片信息在翻译过程中的作用方式,使得部分方法没按照预期为翻译模型带来性能的提升的同时,还难以对方法进行有效的改进。这说明了采用明确方式在神经机器翻译方法中融合图片信息的必要性。
%将图片输入到神经机器翻译模型中具有直接从数据中学习并融合跨模态信息的优点,但也难以明确图片信息的具体作用,因此这类方法可称为隐式跨模态信息融合法。
% 如何解决
为了探究显式跨模态信息融合法是否可行,本文提出了一种基于跨模态文本重构的神经机器翻译方法。
% 具体怎么做
该方法将一个文本重构任务与翻译任务相结合。文本重构模型在训练中将源语言句子中的名词或短语的位置明确地替换为图片中对应的视觉目标,并将该序列输入并重构到完整的源语言句子或目标语言的句子。由于文本重构模型也是端到端的文本生成式模型,因此可以通过参数共享的方式将重构模型编码器或解码器的参数与翻译模型共享,最终达到增强翻译模型的目的。
% 实验效果
实验表明,基于文本重构方法机器翻译模型在测试阶段不需要输入图片的情况下达到与常规方法可比的翻译准确率。并且该方法主要提升了与视觉目标相对应的实体词的翻译准确率。

{\sffamily 2. 基于双向跨模态实体重构的神经机器翻译方法}

%面对问题
显式跨模态信息融合法主要作用在词级或短语级的跨模态信息融合中,使图片信息的作用目标更准确。隐式跨模态信息融合方法主要作用于句子级别的语义融合。上一部分的工作采用的文本重构方法应用了图像到文本单方向的重构,
%如何解决
为了探究如何在显式方法中更进一步利用图片信息,本文提出了一种基于双向跨模态实体重构的神经机器翻译方法,并将其与文本非实体重构方法相结合,实现显式方法与隐式方法的结合。
%具体做法
首先,该方法在图片到文本的重构方向基础上,增加了文本到图像的重构,并且采用更细粒度的实体级重构任务实现双向重构。然后为了结合隐式方法,该方法还增加了文本非实体的重构任务。文本非实体的重构依旧是在文本上下文中采用明确的视觉目标作用方法,但作用到的位置并不是非实体词,因此是一种隐式跨模态信息融合方法。最后将三种重构任务与机器翻译任务相结合达到强化翻译模型的目的。
%文本重构方法在重构的过程中生成了源语言端已经提供的信息。因此本文所提方法抛弃了文本级别的重构,在文本实体和视觉实体之间做双向的实体级重构。并增加了非实体的重构,使图片信息与文本上下文做进一步的信息融合。然后,将以上三种重构任务与翻译任务通过多任务学习的方式结合。
%实验结果
实验表明,该方法在测试阶段不需要输入图片的情况下进一步地提升了机器翻译的质量,双向实体重构与非实体重构的多任务组合方式使显式方法与隐式方法的结合有效。

{\sffamily 3. 基于图文对比对抗训练的神经机器翻译方法}

%面对的问题
前两章提出了图片增强式的神经机器翻译方法,分别利用了显式和隐式的跨模态信息融合方法,通过多任务优化模型参数的方式提升了神经机器翻译的译文质量。此类方法主要针对在模型推理阶段不需要输入图片的情况。然而当文本中出现病句问题时,仍然需要将图片输入到模型中辅助翻译过程,但这类方法普遍存在视觉信息在模型中难以起作用的问题。
%如何解决
针对此问题,本文提出了一种基于图文对比对抗训练的神经机器翻译方法。
%具体做法
为了机器翻译模型更多地关注图片中的信息,需要在模型训练过程增加判断当前输入图片与句子的语义是否一致的功能。因此本文采用了对比学习方法,首先拉近翻译模型的图文对与译文在文本表示空间中的语义关系。然后再向负样本集中加入同为图文对形式的对抗样本。该对抗样本的特点是将原图文对中的图片用随机的错误图片替换。这种方式能使原文同时出现在正样本集和负样本集中。但因为输入的图片不一致,使模型在对比的过程中需要将图片信息融合到文本表示中才能判断样本的正负性,并以此来达到增加图片信息在文本表示中的作用。
%为了拉近双语的语义关系,在编码端增加了图文与目标语言句子之间对比学习。并在负样本集中引入了包含源语言句子+错误图片对抗样本。为了将正负样本区分开,模型需要判断图片信息是否与源语言句子的语义一致。该方法会将图片信息融合到文本的表示中,从而提升视觉信息在模型中的作用程度。
%实验结果
实验结果表明,该方法不仅提升了翻译模型的性能,还提升了模型对视觉信息的敏感度,模型在输入正确图片时的性能明显优于输入错误图片或不输入图片的情况。

综上所述,本文深入研究了图片信息在神经机器翻译模型中的作用方式。本文采用了显式和隐式跨模态信息融合方法,在图片信息增强式和图片信息辅助式的神经机器翻译中,通过对模型结构和模型训练等方面进行设计,增强了图片信息在翻译模型中的作用,提升了模型的翻译性能。
%综上所述,本文旨在设计更好的图片信息融合方法,提升图片信息在神经机器翻译模型中的作用效果。为此,本文设计了显式的跨模态信息融合方法、隐式跨模态信息融合方法以及两种方式相结合的方法。实验表明,本文所提方法能够有效地将图片信息融合到翻译模型中,并为翻译质量带来提升。


% 结论中要体现视觉信息对于翻译带来增益的局限性,只能做简单的信息补全,几乎只能在词级带来语义的增强或补全信息,想要补全整个句子的语义是目前做不到的。

\section{工作展望}

本文以融合图片信息的神经机器翻译方法为主题开展了一系列的研究工作,显著提高了图片信息在神经机器翻译模型中的作用效果。尽管已有大量的研究工作从神经翻译模型的结构设计、模型的训练方式、图片信息的作用方式、目标译文的解码方式等多个方面对融合图片信息的神经机器翻译方法进行改进,但目前还存在需要进一步探讨的问题,主要包括以下方面:

(1)大规模单语数据的应用

目前融合图片信息的神经机器翻译的研究主要在图片描述的翻译任务中展开,所使用的翻译数据不仅需要人为标注的平行翻译句对,还需要为原文或译文匹配与内容高度相关的图片以形成三元组翻译数据。这种情况下,一方面难以收集更大规模的翻译数据,在资源稀缺的情况下,跨模态信息的融合更困难;另一方面,图片描述翻译与现实应用场景中的文本翻译有较大的领域差异,使得现阶段研究成果难以直接投入生产应用。尽管已经有利用大规模视觉语言预训练模型的知识迁移能力和跨模态信息融合能力的神经机器翻译方法,但应用数据仍以图片描述为主。
互联网上的社交媒体或电商平台充斥着大规模图文匹配的单语数据,如何在神经机器翻译中利用这类数据对实际应用具有高度研究价值。

(2)可解释性的研究

从融合图片信息的神经机器翻译的已有研究中不难发现,跨模态信息为翻译带来准确率的提升并不难,但是如何确保翻译模型的增益来自于图片信息,如何明确图片信息在翻译过程中的作用方式,如何在已有的性能增益的基础上更进一步提升翻译质量,都是影响研究进展的关键问题。因此,跨模态信息融合的可解释性研究是确保融合图片信息的神经机器翻译能够长足发展的保证。
%图片怎么起作用,图片起什么作用
%研究人员对模型的进一步改进的依据是什么,这些都是需要可解释性做支撑的。

(3)融合视频信息

相比于图片,融合视频信息的神经机器翻译更进一步地接近人类使用自然语言进行交流的方式。目前已有How2\pcite{sanabria2018how2}和VATEX\pcite{wang2019vatex}等视频形式的多模态翻译数据,主要针对视频中的字幕或语言文字的翻译任务。但相关的研究方法与融合图片的神经机器翻译方法差异并不明显。多采用提取关键帧的方式将文本与关键帧进行跨模态的融合与翻译。视频中主要包含来自文本、语音和图像三个模态的语义信息。如何在文本翻译中,融合来自语音的纠错信息,和来自视频图像的视觉变换信息,是神经机器翻译任务所面临的更大的挑战。
