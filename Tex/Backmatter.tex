%---------------------------------------------------------------------------%
%->> Backmatter
%---------------------------------------------------------------------------%
\chapter[致谢]{致\quad 谢}\chaptermark{致\quad 谢}% syntax: \chapter[目录]{标题}\chaptermark{页眉}
%\thispagestyle{noheaderstyle}% 如果需要移除当前页的页眉
%\pagestyle{noheaderstyle}% 如果需要移除整章的页眉

%此处填写致谢。


\rightline{2023年6月}
\chapter{作者简历及攻读学位期间发表的学术论文与其他相关学术成果}

%\section*{作者简历:}
%××××年××月——××××年××月,在××大学××院(系)获得学士学位。

%××××年××月——××××年××月,在××大学××院(系)获得硕士学位。

%××××年××月——××××年××月,在中国科学院××研究所(或中国科学院大学××院系)攻读博士/硕士学位。

%工作经历:

\section*{已发表(或正式接受)的学术论文:}

{
\setlist[enumerate]{}% restore default behavior
\begin{enumerate}[nosep]
    \item 第一作者. Contrastive Adversarial Training for Multi-modal Machine Translation. ACM Transactions on Asian and Low-Resource Language Information Processing (TALLIP), 2023. Accepted.
    \item 第一作者. Entity-level Cross-modal Learning Improves Multi-modal Machine Translation. In Findings of EMNLP 2021. pages 1067–1080.
    \item 第一作者. 基于跨模态实体信息融合的神经机器翻译方法.自动化学报, 2023, 49(3): 1−11
\end{enumerate}
}

\section*{申请或已获得的专利:}

1. 第二完成人(导师为第一完成人)。多模态机器翻译方法、装置、电子设备和存储介质。202110392717.5。

%\section*{参加的研究项目及获奖情况:}


\cleardoublepage[plain]% 让文档总是结束于偶数页,可根据需要设定页眉页脚样式,如 [noheaderstyle]
%---------------------------------------------------------------------------%
