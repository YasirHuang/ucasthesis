%---------------------------------------------------------------------------%
%->> Backmatter
%---------------------------------------------------------------------------%
\chapter[致谢]{致\quad 谢}\chaptermark{致\quad 谢}% syntax: \chapter[目录]{标题}\chaptermark{页眉}

%\thispagestyle{noheaderstyle}% 如果需要移除当前页的页眉
%\pagestyle{noheaderstyle}% 如果需要移除整章的页眉

%感激casthesis作者吴凌云学长,gbt7714-bibtex-style
%开发者zepinglee,和ctex众多开发者们。若没有他们的辛勤付出和非凡工作,\LaTeX{}菜鸟的我是无法完成此国科大学位论文\LaTeX{}模板ucasthesis的。在\LaTeX{}中的一点一滴的成长源于开源社区的众多优秀资料和教程,在此对所有\LaTeX{}社区的贡献者表示感谢!

%ucasthesis国科大学位论文\LaTeX{}模板的最终成型离不开以霍明虹老师和丁云云老师为代表的国科大学位办公室老师们制定的官方指导文件和众多ucasthesis用户的热心测试和耐心反馈,在此对他们的认真付出表示感谢。特别对国科大的赵永明同学的众多有效反馈意见和建议表示感谢,对国科大本科部的陆晴老师和本科部学位办的丁云云老师的细致审核和建议表示感谢。谢谢大家的共同努力和支持,让ucasthesis为国科大学子使用\LaTeX{}撰写学位论文提供便利和高效这一目标成为可能。

\chapter{作者简历及攻读学位期间发表的学术论文与研究成果}

%\textbf{本科生无需此部分}。

%\section*{作者简历:}

%\subsection*{casthesis作者}

%吴凌云,福建省屏南县人,中国科学院数学与系统科学研究院博士研究生。

%\subsection*{ucasthesis作者}

%莫晃锐,湖南省湘潭县人,中国科学院力学研究所硕士研究生。

\section*{已发表(或正式接受)的学术论文:}

{
\setlist[enumerate]{}% restore default behavior
\begin{enumerate}[nosep]
    \item 第一作者. Entity-level Cross-modal Learning Improves Multi-modal Machine Translation. In Findings of the Association for Computational Linguistics: EMNLP 2021. Association for Computational Linguistics, Punta Cana, Dominican Republic, 1067–1080.
    \item 第一作者. 基于跨模态实体信息融合的神经机器翻译方法.自动化学报, 2023, 49(3): 1−11
    \item 第一作者. Contrastive Adversarial Training for Multi-modal Machine Translation. ACM Transactions on Asian and Low-Resource Language Information Processing (TALLIP), 2023.
\end{enumerate}
}

\section*{申请或已获得的专利:}

1. 第二完成人。多模态机器翻译方法、装置、电子设备和存储介质。202110392717.5。


\cleardoublepage[plain]% 让文档总是结束于偶数页,可根据需要设定页眉页脚样式,如 [noheaderstyle]
%---------------------------------------------------------------------------%
